\documentclass[../Documentation.tex]{subfiles}
\begin{document}
\section{Data}
User Data that can be used for calculations, or for reference for various scenarios.
\subsection{Overview}
All classes (except \code{UserData.java} and \code{UserData\_Tools.java}) inherits any of the major \code{Struct} classes. As mentioned earlier, custom data validity methods can be employed to restrict what can be entered into the objects, as to prevent possible miscalculations.
\dirtree{%
.1 Struct.
.2 SInt.
.3 Resting\_BPM.
.3 Steps.
.3 Lifts (uses the Weight object).
.4 Benches.
.4 Deadlifts.
.4 Squats.
.2 SDouble.
.3 Height.
.3 Weight.
.3 BF.
.3 Activity\_Level.
.2 SString.
.3 Info.
.3 Gender.
}
\code{UserData.java} defines all objects defined above under one file. \code{UserData\_Tools.java} provides a public interface to get and set data, which is used by the \code{Editor.java} and all the classes that inherit \code{Calc\_Abstract.java}.

\code{Lifts.java} is a subclass of \code{SInt.java} that utilizes the \code{Weight.java} class as an object to define the weight of what is being lifted.
\end{document}
