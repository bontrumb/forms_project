\documentclass[../Documentation.tex]{subfiles}
\begin{document}
\section{Structs}
Not a very creative name, but however it provides the foundation for all the data entered into the app.
\subsection{Overview}
The Structs folder contains the fundamental \code{Struct.java}. 
Extending \code{Struct.java},
\begin{itemize}
	\item \code{SInt.java}
	\item \code{SDouble.java}
	\item \code{SString.java}
	\item \code{Calc\_Abstract.java}\ .
\end{itemize}
\code{Struct.java} defines all the get, set methods for the values, the type of data, and what units does the entry have. 

In particular, there are abstract methods to define methods to check valid data entry (for the set method), particularly for any number based data. This is inherited by all classes that inherits any of the four structures above. The data validity method can be customized by adding various different boolean methods that checks the object that is to be entered.

\subsection{\code{Struct.java}}
The core of all data structures in the program. 
\begin{comment}
\begin{itemize} 
	\object{Struct\_Entry}{Object} Holds all the values of the data.
	\object{Struct\_Type}{String} What is the characteristic that describes the object. E.g. Weight, Height, etc.
	\object{Struct\_Units}{String} The units of the type.
	\object{Struct\_Lock}{Boolean} Locks the type and units of the data from modification.
	\item[] \\
	\constructor{Struct}{
\end{itemize}
\end{comment}
\end{document}
